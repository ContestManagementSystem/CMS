\renewcommand{\nomebreve}{Trovare il massimo sottointervallo crescente}
\renewcommand{\titolo}{Max sottointervallo crescente}

\introduzione{}

Ricevete in input una sequenza di $N$ numeri interi.
In output dovete proporre un sottointervallo strettamente crescente di interi estratta dalla sequenza in input che sia massimo in termini di numero di elementi. 


\sezionetesto{Dati di input}
La prima riga del file \verb'input.txt' contiene un numero intero e positivo $N$, la lunghezza della sequenza in input.
La seconda riga contiene, in ordine, gli $N$ numeri della sequenza.

\sezionetesto{Dati di output}
La prima riga del file \verb'output.txt' deve contenere l'intero che rappresenta di quanti numeri risulta composto il massimo sottointervallo crescente.
La seconda riga contiene, i numeri del massimo sotto intervallo crescente.
Nel caso vi sia più di un sottointervallo massimo, cioè con lo stesso numero di elementi, si dia in output solamente il primo incontrato nella sequenza di input.

% Esempi
\sezionetesto{Esempio di input/output}
\esempio{
6

7 2 2 4 3 9
}{
2

2 4}
\esempio{
12

15 24 22 42 51 99 4 3 5 6 7 3
}{
4

22 42 51 99}

% Assunzioni
\sezionetesto{Assunzioni e note}
\begin{itemize}[nolistsep, noitemsep]
  \item $2 \le N \le 1\,000\,000$.
\end{itemize}
  
  \section*{Subtask}
  \begin{itemize}
    \item \textbf{Subtask 0 [10 punti]:} i due esempi del testo.
    \item \textbf{Subtask 1 [10 punti]:} $N = 3$.
    \item \textbf{Subtask 2 [20 punti]:} $N \leq 5$.
    \item \textbf{Subtask 3 [20 punti]:} $N \leq 20$.
    \item \textbf{Subtask 4 [40 punti]:} nessuna restrizione.
  \end{itemize}
  
