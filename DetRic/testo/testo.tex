\documentclass[a4paper,11pt]{article}
\usepackage{nopageno} % visto che in questo caso abbiamo una pagina sola
\usepackage{lmodern}
\renewcommand*\familydefault{\sfdefault}
\usepackage{sfmath}
\usepackage[utf8]{inputenc}
\usepackage[T1]{fontenc}
\usepackage[italian]{babel}
\usepackage{indentfirst}
\usepackage{graphicx}
\usepackage{tikz}
\usepackage{wrapfig}
\newcommand*\circled[1]{\tikz[baseline=(char.base)]{
		\node[shape=circle,draw,inner sep=2pt] (char) {#1};}}
\usepackage{enumitem}
% \usepackage[group-separator={\,}]{siunitx}
\usepackage[left=2cm, right=2cm, bottom=3cm]{geometry}
\frenchspacing

\newcommand{\num}[1]{#1}

% Macro varie...
\newcommand{\file}[1]{\texttt{#1}}
\renewcommand{\arraystretch}{1.3}
\newcommand{\esempio}[2]{
\noindent\begin{minipage}{\textwidth}
\begin{tabular}{|p{11cm}|p{5cm}|}
	\hline
	\textbf{File \file{input.txt}} & \textbf{File \file{output.txt}}\\
	\hline
	\tt \small #1 &
	\tt \small #2 \\
	\hline
\end{tabular}
\end{minipage}
}

\newcommand{\sezionetesto}[1]{
    \section*{#1}
}

\newcommand{\gara}{Informatica A.S 2017/18}

%%%%% I seguenti campi verranno sovrascritti dall'\include{nomebreve} %%%%%
\newcommand{\nomebreve}{}
\newcommand{\titolo}{}

% Modificare a proprio piacimento:
\newcommand{\introduzione}{
    \noindent{\Large \gara{}}
     \newline
      \newline
    \vspace{0.5cm}
    \noindent{\Huge \textbf \titolo{}}    
}


\begin{document}

\renewcommand{\nomebreve}{determinante\_matrice}
\renewcommand{\titolo}{Calcolo Determinante Matrici}

\introduzione{}

Ricevete in input la dimensione $n$ di una matrice quadrata $A$ ed i suoi coefficienti.
Dovete calcolare il suo determinante.

\sezionetesto{Dati di input}
La prima linea del file \verb'input.txt' contiene l'intero $n$: la dimensione della matrice $A$.
Le successive $n$ righe contengono i coefficienti della matrice $A$: queste $n$ linee del file codificano le $n$ righe di $A$ , ogni riga consta quindi di $n$ numeri interi separati da spazi. 
 
\sezionetesto{Dati di output}
Nella prima linea del file \verb'output.txt' occorre riportare la dimensione di $A$: quindi il valore $n$.
La seconda riga deve contenere l'intero che equivale al valore calcolato del determinante di $A$.
   
% Esempi
\sezionetesto{Esempio di input/output}
\esempio{2

1 0

0 1

}{2

 1
 }


\esempio{4

0 3 2 1

1 2 3 4

2 0 1 0

0 -2 1 1

}{4

-35

}

% Assunzioni
\sezionetesto{Assunzioni}
\begin{itemize}[nolistsep, noitemsep]
  \item La dimensione della matrice A in input sarà al massimo $4$.
\end{itemize}
  
  \section*{Subtask}
  \begin{itemize}    
    \item \textbf{Subtask 0 [20 punti]:} $n=1$.
    \item \textbf{Subtask 1 [35 punti]:} $n=2$.
    \item \textbf{Subtask 2 [45 punti]:} nessuna restrizione (oltre quella sul valore massimo della dimensione di A espressa nella sezione di ``Assunzioni'' generali).
  \end{itemize}



\end{document}
