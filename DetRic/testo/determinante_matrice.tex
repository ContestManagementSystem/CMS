\renewcommand{\nomebreve}{determinante\_matrice}
\renewcommand{\titolo}{Calcolo Determinante con Ricorsione}

\introduzione{}

Ricevete in input la dimensione $n$ di una matrice quadrata $A$ ed i suoi coefficienti.
Dovete calcolare il suo determinante.

\sezionetesto{Dati di input}
La prima linea del file \verb'input.txt' contiene l'intero $n$: la dimensione della matrice $A$.
Le successive $n$ righe contengono i coefficienti della matrice $A$: queste $n$ linee del file codificano le $n$ righe di $A$ , ogni riga consta quindi di $n$ numeri interi separati da spazi. 
 
\sezionetesto{Dati di output}
Nella prima linea del file \verb'output.txt' occorre riportare la dimensione di $A$: quindi il valore $n$.
La seconda riga deve contenere l'intero che equivale al valore calcolato del determinante di $A$.
   
% Esempi
\sezionetesto{Esempio di input/output}
\esempio{2

1 0

0 1

}{2

 1
 }


\esempio{4

0 3 2 1

1 2 3 4

2 0 1 0

0 -2 1 1

}{4

-35

}

% Assunzioni
\sezionetesto{Assunzioni}
\begin{itemize}[nolistsep, noitemsep]
  \item La dimensione della matrice A in input sarà al massimo $6$.
\end{itemize}
  
  \section*{Subtask}
  \begin{itemize}    
    \item \textbf{Subtask 0 [20 punti]:} $n=3$.
    \item \textbf{Subtask 1 [20 punti]:} $n=4$.
    \item \textbf{Subtask 2 [60 punti]:} nessuna restrizione (oltre quella sul valore massimo della dimensione di A espressa nella sezione di ``Assunzioni'' generali).
  \end{itemize}

